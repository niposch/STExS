\documentclass[11pt]{article}
\usepackage{threeparttable}             % tables with footnotes, capions all the same width
\usepackage{dcolumn}                    % decimal-aligned tabular math columns
\usepackage{multirow}                   % Allow table cells to span multiple rows
\usepackage{booktabs}                   % Formatting options for publication-quality tables
\usepackage{ltxtable}                   % long tabularx
\usepackage{colortbl}
\usepackage[table]{xcolor}
\usepackage{tabularx}



\begin{document}






\begin{table}[h]
\rowcolors{1}{gray}{white}
\begin{tabularx}{\textwidth}{|X|X|}
\hline
\multicolumn{2}{|c|}{Parson}\\ \hline
Disassembles Code& ParsonElement  \\  \hline
Memorizes Orginal Code Order & ParsonExersiseGrader\\ \hline
Shuffles Code & \\ \hline
\end{tabularx}

\end{table}


\begin{table}[h]
\rowcolors{1}{gray}{white}
\begin{tabularx}{\textwidth}{|X|X|}
\hline
\multicolumn{2}{|c|}{ParsonElement}\\ \hline
Contains code fragment& \\  \hline
 & \\ \hline
 & \\ \hline
\end{tabularx}

\end{table}


\begin{table}[h]
\rowcolors{1}{gray}{white}
\begin{tabularx}{\textwidth}{|X|X|}
\hline
\multicolumn{2}{|c|}{Syntax}\\ \hline
Compares output with a set solution& CodeRunner \\  \hline
 &  \\ \hline
 & \\ \hline
\end{tabularx}

\end{table}


\begin{table}[h]
\rowcolors{1}{gray}{white}
\begin{tabularx}{\textwidth}{|X|X|}
\hline
\multicolumn{2}{|c|}{Ausgabe des Codes}\\ \hline
Compares output prediction with the actual solution&  \\  \hline
 &  \\ \hline
 & \\ \hline
\end{tabularx}

\end{table}

\begin{table}[h]
\rowcolors{1}{gray}{white}
\begin{tabularx}{\textwidth}{|X|X|}
\hline
\multicolumn{2}{|c|}{Kapitel}\\ \hline
Verwaltung der Aufgaben eines Kapitels&Aufgabe   \\  \hline
\end{tabularx}

\end{table}

\begin{table}[h]
\rowcolors{1}{gray}{white}
\begin{tabularx}{\textwidth}{|X|X|}
\hline
\multicolumn{2}{|c|}{Aufgabe}\\ \hline
Enthält eine Aufgabe von einem bestimmten Aufgabentyp&Aufgabentyp   \\  \hline
\end{tabularx}

\end{table}

\end{document}
\end{document}
