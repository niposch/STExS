\documentclass[11pt]{article}
\usepackage{threeparttable}             % tables with footnotes, capions all the same width
\usepackage{dcolumn}                    % decimal-aligned tabular math columns
\usepackage{multirow}                   % Allow table cells to span multiple rows
\usepackage{booktabs}                   % Formatting options for publication-quality tables
\usepackage{ltxtable}                   % long tabularx
\usepackage{colortbl}
\usepackage[table]{xcolor}
\usepackage{tabularx}



\begin{document}






\begin{table}[h]
\rowcolors{1}{gray}{white}
\begin{tabularx}{\textwidth}{|X|X|}
\hline
\multicolumn{2}{|c|}{ParsonTask}\\ \hline
Disassembles Code& ParsonElement  \\  \hline
Memorizes Orginal Code Order & ParsonExersiseGrader\\ \hline
Shuffles Parson Elements & \\ \hline
\end{tabularx}

\end{table}


\begin{table}[h]
\rowcolors{1}{gray}{white}
\begin{tabularx}{\textwidth}{|X|X|}
\hline
\multicolumn{2}{|c|}{ParsonElement}\\ \hline
Contains code fragment& \\  \hline
\end{tabularx}

\end{table}


\begin{table}[h]
\rowcolors{1}{gray}{white}
\begin{tabularx}{\textwidth}{|X|X|}
\hline
\multicolumn{2}{|c|}{SyntaxTaskGrader}\\ \hline
Compares output with a predefined solution& CodeRunner \\  \hline
 &  SourceCode\\ \hline
 & SyntaxTaskSolution\\ \hline
\end{tabularx}

\end{table}

\begin{table}[h]
\rowcolors{1}{gray}{white}
\begin{tabularx}{\textwidth}{|X|X|}
\hline
\multicolumn{2}{|c|}{SyntaxTaskSolution}\\ \hline
speichert Lösungsinformationen bzgl. SyntaxTasks& \\  \hline
\end{tabularx}

\end{table}




\begin{table}[h]
\rowcolors{1}{gray}{white}
\begin{tabularx}{\textwidth}{|X|X|}
\hline
\multicolumn{2}{|c|}{CodeResultPredictionTaskGrader}\\ \hline
Compares output prediction with the actual solution& GradingResult  \\  \hline
Creates GradingResult &  \\ \hline
\end{tabularx}

\end{table}

\begin{table}[h]
\rowcolors{1}{gray}{white}
\begin{tabularx}{\textwidth}{|X|X|}
\hline
\multicolumn{2}{|c|}{AdminService}\\ \hline
Auflistung aller Kurse & Module  \\  \hline
Verwaltung aller Kurse (Adminrecht) & ApplicationUser \\ \hline
Nutzerverwaltung & \\ \hline
\end{tabularx}
\end{table}

\begin{table}[h]
\rowcolors{1}{gray}{white}
\begin{tabularx}{\textwidth}{|X|X|}
\hline
\multicolumn{2}{|c|}{Module}\\ \hline
Auflistung aller Kapitel & Chapter  \\  \hline
Verwaltung aller Kapitel (Adminrecht) & \\ \hline
\end{tabularx}
\end{table}

\begin{table}[h]
\rowcolors{1}{gray}{white}
\begin{tabularx}{\textwidth}{|X|X|}
\hline
\multicolumn{2}{|c|}{Chapter}\\ \hline
Verwaltung der Aufgaben eines Kapitels & Exercise   \\  \hline
\end{tabularx}

\end{table}

\begin{table}[h]
\rowcolors{1}{gray}{white}
\begin{tabularx}{\textwidth}{|X|X|}
\hline
\multicolumn{2}{|c|}{Exercise}\\ \hline
Enthält eine Aufgabe von einem bestimmten Aufgabentyp & ExerciseType   \\  \hline
\end{tabularx}

\end{table}


\begin{table}[h]
\rowcolors{1}{gray}{white}
\begin{tabularx}{\textwidth}{|X|X|}
\hline
\multicolumn{2}{|c|}{ChapterGrader}\\ \hline
Berechnet automatisiert mittels der ExerciseGrader die im Chapter erreichte Punktzahl und speichert diese als ChapterGradingResult & ChapterGradingResult \\  \hline
& ExerciseGrader (ParsonExerciseGrader, SyntaxExerciseGrader etc.)\\ \hline
 & Chapter\\ \hline
 & Exercise  \\
\end{tabularx}
\end{table}

\begin{table}[h]
\rowcolors{1}{gray}{white}
\begin{tabularx}{\textwidth}{|X|X|}
\hline
\multicolumn{2}{|c|}{ChapterGradingResult}\\ \hline
Speichert Bewertungsinformationen für ApplicationUser und Chapter& ApplicationUser, Chapter \\  \hline
Kann im Nachgang von Kursleitern bearbeitet werden& \\ \hline
& \\
\end{tabularx}
\end{table}

%
\begin{table}[h]
\rowcolors{1}{gray}{white}
\begin{tabularx}{\textwidth}{|X|X|}
\hline
\multicolumn{2}{|c|}{ExerciseGrader}\\ \hline
Ein Interface, welches die Aufgabenbewertung vereinheitlicht & ParsonExerciseGrader  \\  \hline
Erstellt GradingResult, wenn möglich mit Bewertungsergebnissen &SyntaxExerciseGrader \\ \hline
Hält Notwendigkeit von manueller Bewertung im GradingResult fest, falls autom. Korrektur nicht ausreichend für AufgabenTyp ist & GradingResult\\ \hline
\end{tabularx}
\end{table}

\begin{table}[h]
\rowcolors{1}{gray}{white}
\begin{tabularx}{\textwidth}{|X|X|}
\hline
\multicolumn{2}{|c|}{ParsonSolution}\\ \hline
enthält Daten über die richtige Antwort auf ein ParsonTask & ParsonTask \\ \hline
\end{tabularx}
\end{table}

\begin{table}[h]
\rowcolors{1}{gray}{white}
\begin{tabularx}{\textwidth}{|X|X|}
\hline
\multicolumn{2}{|c|}{ParsonExerciseGrader}\\ \hline
Erstellt GradingResult anhand von Nutzerabgabe und ParsonSolution &ParsonSolution   \\ \hline
&GradingResult \\ \hline
\end{tabularx}
\end{table}


\begin{table}[h]
\rowcolors{1}{gray}{white}
\begin{tabularx}{\textwidth}{|X|X|}
\hline
\multicolumn{2}{|c|}{UnitTest}\\ \hline
Testet richtige Funktionalität von abgegebenen Coding-Aufgaben mithilfe von Testdaten&SourceCode\\  \hline
\end{tabularx}
\end{table}

\begin{table}[h]
\rowcolors{1}{gray}{white}
\begin{tabularx}{\textwidth}{|X|X|}
\hline
\multicolumn{2}{|c|}{CodeRunner}\\ \hline
Übernimmt Kommunikation mit Code-Ausführungs-Tool für das Testen abgegebener Coding-Aufgaben&UnitTest  \\  \hline
&SourceCode \\ \hline
\end{tabularx}
\end{table}


\begin{table}[h]
\rowcolors{1}{gray}{white}
\begin{tabularx}{\textwidth}{|X|X|}
\hline
\multicolumn{2}{|c|}{SourceCode}\\ \hline
Persistenz von Usercode& \\  \hline
\end{tabularx}
\end{table}

%Jonathan
\begin{table}[h]
\rowcolors{1}{gray}{white}
\begin{tabularx}{\textwidth}{|X|X|}
\hline
\multicolumn{2}{|c|}{ApplicationUser}\\ \hline
Manages user data& LoginInformation  \\  \hline
&RegistrationInformation \\ \hline
 &ChapterGrading \\ \hline
\end{tabularx}
\end{table}

\begin{table}[h]
\rowcolors{1}{gray}{white}
\begin{tabularx}{\textwidth}{|X|X|}
\hline
\multicolumn{2}{|c|}{LoginInformation}\\ \hline
Contains login data& \\  \hline
\end{tabularx}
\end{table}

\begin{table}[h]
\rowcolors{1}{gray}{white}
\begin{tabularx}{\textwidth}{|X|X|}
\hline
\multicolumn{2}{|c|}{RegistrationInformation}\\ \hline
Contains registration data& \\  \hline
\end{tabularx}
\end{table}

\begin{table}[h]
\rowcolors{1}{gray}{white}
\begin{tabularx}{\textwidth}{|X|X|}
\hline
\multicolumn{2}{|c|}{UserService}\\ \hline
Führt Login, Registrierung etc. & User, LoginInformation, RegistrationInformation\\  \hline
\end{tabularx}
\end{table}    

%Bruno
\begin{table}[h]
\rowcolors{1}{gray}{white}
\begin{tabularx}{\textwidth}{|X|X|}
\hline
\multicolumn{2}{|c|}{Finde den Bug}\\ \hline
Locate and fix faulty Code&Code Runner \\  \hline
&Exercise Grader \\ \hline
\end{tabularx}
\end{table}

\begin{table}[h]
\rowcolors{1}{gray}{white}
\begin{tabularx}{\textwidth}{|X|X|}
\hline
\multicolumn{2}{|c|}{NaturalLanguageDescriptionTask}\\ \hline
speichert und verwalted Daten bzgl. der nat. sprachl. Aufgaben& \\  \hline
\end{tabularx}
\end{table}

\begin{table}[h]
\rowcolors{1}{gray}{white}
\begin{tabularx}{\textwidth}{|X|X|}
\hline
\multicolumn{2}{|c|}{ProgrammingTask}\\ \hline
speichert und verwalted Daten bzgl. der ProgrammingTask & \\  \hline
\end{tabularx}
\end{table}

\begin{table}[h]
\rowcolors{1}{gray}{white}
\begin{tabularx}{\textwidth}{|X|X|}
\hline
\multicolumn{2}{|c|}{ProgrammingTaskGrader}\\ \hline
Bewertet automatisch Programmieraufgaben & ProgrammingTask\\  \hline
\end{tabularx}
\end{table}

\begin{table}[h]
\rowcolors{1}{gray}{white}
\begin{tabularx}{\textwidth}{|X|X|}
\hline
\multicolumn{2}{|c|}{GradingResult}\\ \hline
Speichert Bewertungsinformationen für ApplicationUser und Exercise& ApplicationUser, Exercise \\  \hline
Kann im Nachgang von Kursleitern bearbeitet werden& \\ \hline
\end{tabularx}
\end{table}

\begin{table}[h]
\rowcolors{1}{gray}{white}
\begin{tabularx}{\textwidth}{|X|X|}
\hline
\multicolumn{2}{|c|}{ApplicationDbContext} \\ \hline
 Abstrahiert Anwendung von verwendeter Datenbanktechnologie.&z.B. Modul, Chapter, SyntaxTaskSolution etc. \\   \hline
Übernimmt Kommunikation mit der Datenbank & (jede Klasse die in der Datenbank zu einer Tabelle wird)\\  \hline
Automatisierte Erstellung der Migrationen & \\ \hline
\end{tabularx}
\end{table}

\end{document}
\end{document}
