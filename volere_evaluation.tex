\documentclass{article}

\usepackage{multicol}
\usepackage[dvipsnames]{xcolor}
\usepackage{enumitem}
\usepackage{tcolorbox}

\newenvironment{myreq}[1]{%
\setlist[description]{font=\normalfont\color{darkgray}}%
\begin{tcolorbox}[colframe=black,colback=white, sharp corners, boxrule=1pt]%
\bfseries\color{blue}%
\begin{description}#1}%
{\end{description}\end{tcolorbox}}

\newcommand{\threeinline}[3]{\begin{multicols}{3}#1 #2 #3\end{multicols}}
\newcommand{\twoinline}[2]{\begin{multicols}{2}#1 #2\end{multicols}}

\newcommand{\reqno}{\item[Requirement \#:]}
\newcommand{\reqtype}{\item[Requirement Type:]}
\newcommand{\reqevent}{\item[Event/BUC/PUC \#:]}
\newcommand{\reqdesc}{\item[Description:]}
\newcommand{\reqrat}{\item[Rationale:]}
\newcommand{\reqorig}{\item[Originator:]}
\newcommand{\reqfit}{\item[Fit Criterion:]}
\newcommand{\reqsatis}{\item[Customer Satisfaction:]}
\newcommand{\reqdissat}{\item[Customer Dissatisfaction:]}
\newcommand{\reqprio}{\item[Priority:]}
\newcommand{\reqconf}{\item[Conflicts:]}
\newcommand{\reqmater}{\item[Materials:]}
\newcommand{\reqhist}{\item[History:]}

\begin{document}
%Feedback von User
\begin{myreq}
  \threeinline
    {\reqno 75}
    {\reqtype 9}
    {\reqevent 7.9}
  \reqdesc Feedback-Funktion für den Nutzer
  \reqrat Besseres Einblick zu dem Lernprozess des Nutzers erhalten. Schwächen sollen effizienter erkannt und behoben werden
  \reqorig Bruno Standke
  \reqfit longer text that needs more than one line longer text that needs more than one line  %!!!!!
  \twoinline
    {\reqsatis 2}
    {\reqdissat 1}
  \twoinline
  {\reqprio 1}
  {\reqconf N.A.}
  \reqmater https://www.barclays.co.uk/content/dam/documents/ \newline
business/business-insight/feedback-economy.pdf
  \reqhist 06.11.2022: Bruno Standke
\end{myreq}

%Feedback Admin
\begin{myreq}
  \threeinline
    {\reqno 75}
    {\reqtype 9}
    {\reqevent 7.9}
  \reqdesc Kommentarfunktion von Moderator für Nutzer
  \reqrat Nach beenden einer Aufgabe kann der Nutzer von einen Moderator individuelles Feedback erhalten. Diese sollen den Nutzer auf Schwächen hinweisen und Lösungsansätze anbieten
  \reqorig Bruno Standke
  \reqfit Nutzer kann angeben, ob Feedback hilfreich war
  \twoinline
    {\reqsatis 3}
    {\reqdissat 1}
  \twoinline
  {\reqprio 4}
  {\reqconf N.A.}
  \reqmater https://www.microsoft.com/en-us/research/wp-content/uploads/2016/02/bosu2015useful.pdf
  \reqhist 07.11.2022: Bruno Standke
\end{myreq}
%automatisiertes Auswerten von Ergebnissen (Kompilierung+Korrektheit)
\begin{myreq}
  \threeinline
    {\reqno 75}
    {\reqtype 9}
    {\reqevent 7.9}
  \reqdesc Automatisierte Bewertung von Lösungen
  \reqrat Lösungen des Nutzers sollen vom System selbständig kompiliert und ausgewertet werden
  \reqorig Dominik Gorgosch
  \reqfit Code soll in unter 5 Sekunden kompiliert und ausgeführt werden. Ausgabe wird mit einer Musterlösung verglichen
  \twoinline
    {\reqsatis 5}
    {\reqdissat 5}
  \twoinline
  {\reqprio +inf}
  {\reqconf N.A.}
  \reqmater N.A.
  \reqhist 07.11.2022: Bruno Standke
\end{myreq}
\end{document}