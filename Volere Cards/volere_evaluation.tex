\documentclass{article}

\usepackage{multicol}
\usepackage[dvipsnames]{xcolor}
\usepackage{enumitem}
\usepackage{tcolorbox}
\usepackage{hyperref}

\newcounter{requirementCounter} % create a new counter, called 'mycounter'
% default def'n of '\themycounter' is '\arabic{mycounter}'

%% command to increment 'mycounter' by 1 and to display its value:
\newcommand\insertRequirementCount{\stepcounter{requirementCounter}\therequirementCounter}

\newenvironment{myreq}[1]{%
\setlist[description]{font=\normalfont\color{darkgray}}%
\begin{tcolorbox}[colframe=black,colback=white, sharp corners, boxrule=1pt]%
\bfseries\color{blue}%
\begin{description}#1}%
{\end{description}\end{tcolorbox}}
\newcommand{\threeinline}[3]{\begin{multicols}{3}#1 #2 #3\end{multicols}}
\newcommand{\twoinline}[2]{\begin{multicols}{2}#1 #2\end{multicols}}

\newcommand{\reqno}{\item[Requirement \#: ]\insertRequirementCount{}}
\newcommand{\reqtype}{\item[Req. Type:]}
\newcommand{\reqevent}{\item[Sprint \#:]}
\newcommand{\reqdesc}{\item[Description:]}
\newcommand{\reqrat}{\item[Rationale:]}
\newcommand{\reqorig}{\item[Originator:]}
\newcommand{\reqfit}{\item[Fit Criterion:]}
\newcommand{\reqsatis}{\item[Customer Satisfaction:]}
\newcommand{\reqdissat}{\item[Customer Dissatisfaction:]}
\newcommand{\reqprio}{\item[Priority:]}
\newcommand{\reqconf}{\item[Conflicts:]}
\newcommand{\reqmater}{\item[Materials:]}
\newcommand{\reqhist}{\item[History:]}

\begin{document}
Requirements types: Functional (F), Non-Functional (NF)
%Feedback von User
\begin{myreq}
  \threeinline
  {\reqno}
  {\reqtype 9}
  {\reqevent 7.9}
  \reqdesc Feedback-Funktion für den Nutzer
  \reqrat Besseres Einblick zu dem Lernprozess des Nutzers erhalten. Schwächen sollen effizienter erkannt und behoben werden
  \reqorig Bruno Standke
  \reqfit longer Kommentarbox am Ende jeder Aufgabe  %!!!!!
  \twoinline
  {\reqsatis 2}
  {\reqdissat 1}
  \twoinline
  {\reqprio 1}
  {\reqconf N.A.}
  \reqmater \url{https://www.barclays.co.uk/content/dam/documents/business/business-insight/feedback-economy.pdf}
  \reqhist 06.11.2022: Bruno Standke
\end{myreq}

%Feedback Admin
\begin{myreq}
  \threeinline
  {\reqno}
  {\reqtype 9}
  {\reqevent 7.9}
  \reqdesc Kommentarfunktion von Moderator für Nutzer
  \reqrat Nach beenden einer Aufgabe kann der Nutzer von einen Moderator individuelles Feedback erhalten. Diese sollen den Nutzer auf Schwächen hinweisen und Lösungsansätze anbieten
  \reqorig Bruno Standke
  \reqfit Nutzer kann angeben, ob Feedback hilfreich war
  \twoinline
  {\reqsatis 3}
  {\reqdissat 1}
  \twoinline
  {\reqprio 4}
  {\reqconf N.A.}
  \reqmater \url{https://www.microsoft.com/en-us/research/wp-content/uploads/2016/02/bosu2015useful.pdf}
  \reqhist 07.11.2022: Bruno Standke
\end{myreq}
%automatisiertes Auswerten von Ergebnissen (Kompilierung+Korrektheit)
\begin{myreq}
  \threeinline
  {\reqno}
  {\reqtype Technical Requirement}
  {\reqevent 7.9}
  \reqdesc Automatisierte Bewertung von Lösungen
  \reqrat Lösungen des Nutzers sollen vom System selbständig kompiliert und ausgewertet werden
  \reqorig Dominik Gorgosch
  \reqfit Code soll in unter 5 Sekunden kompiliert und ausgeführt werden. Ausgabe wird mit einer Musterlösung verglichen
  \twoinline
  {\reqsatis 5}
  {\reqdissat 5}
  \twoinline
  {\reqprio +inf}
  {\reqconf N.A.}
  \reqmater N.A.
  \reqhist 07.11.2022: Bruno Standke
\end{myreq}


\begin{myreq}
  \threeinline
  {\reqno}
  {\reqtype F}
  {\reqevent KW45}
  \reqdesc Nutzerkonten werden durch das System eindeutig identifiziert
  \reqrat es muss für bestimmte Funktionen ein Nutzerkonto eindeutig identifiziert werden können
  \reqorig Entwickler
  \reqfit jeder Nutzer muss bei der Accounterstellung mit einer eindeutigen und unveränderlichen Id versehen werden. Diese muss vom System gespeichert werden.
  \twoinline
  {\reqsatis 1}
  {\reqdissat 1}
  \twoinline
  {\reqprio +inf}
  {\reqconf N.A.}
  \reqmater UUIDv4 kann hierfür verwendet werden \url{https://en.wikipedia.org/wiki/Universally_unique_identifier}
  \reqhist 11.11.2022: erstellt von Nick
\end{myreq}
\begin{myreq}
  \threeinline
  {\reqno}
  {\reqtype NF}
  {\reqevent KW45}
  \reqdesc Kritische Systemfehler(z.B. Abgabe konnte nicht gespeichert werden) werden dem Nutzer mitgeteilt
  \reqrat Nutzer müssen auf Fehlerzustände hingewiesen werden
  \reqorig Endnutzer
  \reqfit Wenn Fehler Auftreten, muss dies durch die Anwendung erkannt und dem Nutzer mitgeteilt werden
  \twoinline
  {\reqsatis 2}
  {\reqdissat 6}
  \twoinline
  {\reqprio 8}
  {\reqconf N.A.}
  \reqmater \url{https://uxplanet.org/how-to-write-good-error-messages-858e4551cd4}
  \reqhist 11.11.2022: erstellt von Nick
\end{myreq}


%Jonathans Volere Snow Cards
\begin{myreq}
  \threeinline
  {\reqno 01}
  {\reqtype 2}
  {\reqevent 3 (6)}
  \reqdesc Code Editor mit Syntax Highlighting von Java, Python
  \reqrat Syntax / Programmieraufgaben mit Code Editor mit Syntax Highlighting realistisch umsetzbar
  \reqorig Nutzer, die Syntax / Programmieraufgaben lösen müssen
  \reqfit gehighlighteten Code mit Code in IDEs vergleichen
  \twoinline
  {\reqsatis 5}
  {\reqdissat 9}
  \twoinline
  {\reqprio 8}
  {\reqconf N.A.}
  \reqmater -
  \reqhist 11.11.2022 Anforderung erstellt
\end{myreq}

\begin{myreq}
  \threeinline
  {\reqno 02}
  {\reqtype 2}
  {\reqevent 9.9}
  \reqdesc Code Editor muss in unter 0.125 einen Tastendruck registrieren
  \reqrat längere Reaktionszeiten sind nicht Nutzerfreundlich und sorgen für Unzufriedenheit
  \reqorig Nutzer
  \reqfit Zeitmessung im Code
  \twoinline
  {\reqsatis 5}
  {\reqdissat 9}
  \twoinline
  {\reqprio 8}
  {\reqconf N.A.}
  \reqmater -
  \reqhist 11.11.2022 Anforderung erstellt
\end{myreq}



\end{document}
